\section {Tímaháðar líkindadreifingar (stationary distributions)}


Tímaháðar líkindadreifingar er í raun aðferð til að skoða hegðun dreifinga fyrir eitthvað kerfi, aðallega til að sjá hvort að dreifing geti haldið sér yfir mörg tímabil. Helst er þar talað um stationary Markov keðjur þar sem að skoðuð eru tilfelli í ótímaháðu umhverfi til að draga ályktun um hegðun kerfis þegar að tími skiptir máli. Þá er gert ráð fyrir því að stationary eiginleikar haldist yfir öll möguleg tímaástönd. Einnig er oft talað um jaðardreyfingu á kyrrstæðu kerfi eða stöðuðu tímabili sem stationary distribution. Í síðasta lagi nær það utan um joint probability distribution á stöðnuðu tímabili.\\

Skilyrði fyrir stationary distribution í Markov keðju er að allavega eitt jákvætt endurkvæmt ástand sé til (positive recurrent), veit ekki alveg hvort að okkar gagnagrunnur mundi falla undir þetta. Hvort að einungis þurfi að vera ein endurtekning eða kerfisbundnar endurtekningar í keðjunni.\\

Notagildi er ekki beint ein afmörkuð notkun heldur meira bara skilyrði fyrir notkun á öðrum líkindadreifingum til þess að sjá til þess að þær haldist óbreyttar yfir tímabil.  Við munum þurfa að athuga hvort að okkar gagnagrunnur sé með stationary distribution.
