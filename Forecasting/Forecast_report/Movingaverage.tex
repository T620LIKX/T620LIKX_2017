\section{Aðferðin}
Hreyfið meðaltal eða Moving Average er aðferð sem er nokkurskonar milli aðferð á milli tveggja mjög öfgakenndra spálíkana. Það fyrr er hreint meðaltal sem telur besta forspáar gildið vera meðaltal af öllum þeim tímabilum (vikum, mánuðum eða árum) liðnum þar til nú. Það seinna er hið svokallaða Naive eða Constant módel sem telur besta forspáargildið fyrir morgundaginn það sem gerðist í dag og fyrri atburðir ónauðsynlegir. Hugmyndin á bakvið Moving Average er blanda af þessum tveim þar sem meðaltal er tekið af ákveðnum gildum í fortíðinni til þess að spá um fram í tímann. Þetta er aðferð sem er algeng og er mjög víða notuð, sem dæmi má nefna að hreyfið meðaltal og vegið hreyfið meðaltal notað í fjármálageiranum þegar fylgst er með þróun verða. 

\section{Hvernig virkar hún}
Til eru fleiri en ein útfærsla á þessari aðferð en sú einfaldasta er að taka venjulegt meðaltal af nýjustu m gildum tímabils, þar sem m er heiltala. Þetta er það sem kallast Simple Moving Average og er notað til þess að spá fyrir gildi Y fyrir tímabil t+1. Hreyfið meðaltal notar eftirfarandi formúlu.
$$Y_{x+1} = \frac{Y_t + Y_{t-1} + ... + Y_{t-1+m}}{m}$$
Það gefur auga leið að hvert gildi í spálíkani hefur því vægi 1/m og því fleiri gildi sem notuð eru í líkani því mýkri verða gildi sem koma úr líkaninu. 


\section{Forsendur}
Hreyfið meðaltal er líkan sem krefst mikils magn af gögnum og hentar yfirleitt best þar sem sveiflur í gögnum eru minnisháttar, þó er hægt að nota hana þar sem árstíðabundin sveifla er. Til þess að líkanið taki við sér við slíkar aðstæður þarf þó að minnka m, við það erum við þó að fórna stöðugleika spánnar og líkanið verður kvikulla.  
Helstu kostir þessa líkans eru að það er einfalt og auðskiljanlegt og gefur góðar niðurstöður við stöðugar aðstæður. Helsti galli þessarar aðferðar er að hætta er á töf þegar kemur að sveiflum þegar valið er of stórt m, að líkanið taki seinna við sér við breytingum heldur en önnur líkön. Þó er hægt að bæta upp fyrir þessi töf með vigtuðu hreyfðu meðaltali en fjallað verður um það seinna. \cite{Averagingandsmoothingmodels}