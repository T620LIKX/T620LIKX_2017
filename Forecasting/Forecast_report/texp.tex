\section {Þreföld veldisjöfnun}


Þegar að spá á fyrir með veldisjöfnun og gögninn sem á að nota innahalda línulega hneigð og árstíðarsveiflu þá þarf að beita þrefaldri veldisjöfnun.  Til að geta beit þessari spá aðferð þarf að lámarki að hafa gögn fyrir tvær árstíðir.

\subsection{Skilgreiningar á táknum}
	$F_(t+m)=$ spá fyrir tímabilið  t+m \\
	$S_t =$  veldisjöfnun fyrir tímabil t\\ 	
	$b_t =$ línuleg hneigð jöfnin \\
	$I_t =$ árstíðarsveiflujöfnn jöfnun fyrir tímabil ts \\
	$y =$ gögn\\
	$S =$ veldisjöfnuð gögn \\
	$I =$	árstíðar stuðull \\
	$b =$ línuleg hneigð bendill \\
	$m =$ fasti	\\
	$t =$ tíma bendill \\
	$L =$ fjöldi lotna í árstíð \\
	$\alpha =$ fasti \\
	$\beta =$ fasti \\
	$\gamma =$ fasti \\
	$A_p =$ meðaltal fyrir lotu í árstíð\\


\subsection{Jöfnur}
	
	
	$$S_t = \alpha\dfrac{y_t}{I_{t - L}} + (1-\alpha)(S_{t-1}+ b_{t-1} ) $$
	$$b_t = \gamma  (S_{t}-S_{t-1})+(1- \gamma) + (1 - \gamma)b_{t-1}$$
	$$I_t = \beta \dfrac{y_t}{S_t} +(1-\beta)I_{t-L} $$
	$$F_t+m = (S_{t} +mb_{t})I_{t-L+m}$$
	$$b= \dfrac{1}{L}(\dfrac{y_{L+1}-y_1}{L} + \dfrac{y_{L+2}-y_2}{L}+......+ \dfrac{y_{L+L}-y_L}{L} ) $$
	$$A_p = \dfrac{\sum_{i=1}^{t=10} y_i}{L} $$