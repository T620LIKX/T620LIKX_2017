\section {Vegið Hreyfið Meðaltal}
 
Vegið hreyfið meðaltal er aðferð sem er mjög sambærileg og hreyfið meðaltal að mörgu leyti. Munurinn á þessum tveim aðferðum liggur helst í því að í þessari aðferð er sett aukið vægi á ákveðin gildi og þá langoftast er sett aukið vægi á nýjustu gildin. Þessi aðferð hentar einnig betur til að spá fyrir um trend. Val á vægisstuðlum er stundum byggt á reynslu en einnig eru til reikningsaðferðir til að finna þessa stuðla. Til þess að geta notað þessa aðferð þarf að velja eitthvert tímabil sem er svo skipt niður, hvernig tímabilinu er skipt niður fer alfarið eftir upplýsingum og hvað verið er að skoða. Einnig þarf að hafa upplýsingar um þetta tímabil, sem dæmi sölutölur eða innhringingar. \cite{WMA}

\subsection{Skilgreiningar á táknum}

	
	$WMA =$ Vegið hreyfið meðaltal \\
	$d_i =$ Tímabilið sem skoðað er. Teljarinn getur verið til 	dæmis dagar eða mánuðir \\
	$S_i =$ Upplýsingar um til dæmis sölutölur eða innhringar. Teljarinn getur til dæmis verið dagar eða mánuðir \\ 
	$D =$ Summan af fjölda tímabila. \\
	$WT_i =$ Stuðullinn fyrir hvert tímabil. \\
	$i =$ Teljari frá 1 og upp í óendanlegt. \\


\subsection{Jöfnur}
	
	
	$$WT_i = \frac{d_i}{D} $$ \\

	$$WMA = \sum_{n=1}^{\infty}WT_i*S_i$$ \\
